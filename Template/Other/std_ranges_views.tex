\texttt{std::views} 是\textbf{惰性求值}的,只会在遍历时计算;并且\textbf{不复制}数据

\begin{lstlisting}[style=C++]
vector<int> nums{1, 2, 3, 4, 5, 6};  // 假设有这个数组
\end{lstlisting}

过滤 \texttt{views::filter}

\begin{lstlisting}[style=C++]
auto even = nums | views::filter(int n { return n % 2 == 0; });
for (int n : even) std::cout << n << " ";  // 输出: 2 4 6
\end{lstlisting}

转换 \texttt{views::transform}

\begin{lstlisting}[style=C++]
auto squares = nums | views::transform(int n { return n * n; });
for (int n : squares) std::cout << n << " ";  // 输出: 1 4 9 16 25 36
\end{lstlisting}

取前 $N$ 个 / 最后 $N$ 个 \texttt{views::take / views::take\_{}last}

\begin{lstlisting}[style=C++]
auto first3 = nums | views::take(3);  // 输出: 1 2 3
auto last3 = nums | views::take_last(3);  // 输出: 4 5 6
\end{lstlisting}

丢弃前 $N$ 个 / 最后 $N$ 个 \texttt{views::drop / views::drop\_{}last}

\begin{lstlisting}[style=C++]
auto without_first2 = nums | views::drop(2);  // 输出: 3 4 5 6
auto without_last2 = nums | views::drop_last(2);  // 输出: 1 2 3 4
\end{lstlisting}

管道操作符 \texttt{|}

\begin{lstlisting}[style=C++]
// 链式操作:过滤偶数 -> 平方 -> 取前2个
auto result = nums 
    | views::filter(int n { return n % 2 == 0; })
    | views::transform(int n { return n * n; })
    | views::take(2);
for (int n : result) std::cout << n << " ";  // 输出: 4 16
\end{lstlisting}

反转 \texttt{views::reverse}

\begin{lstlisting}[style=C++]
auto reversed = nums | views::reverse;  // 输出: 6 5 4 3 2 1
\end{lstlisting}

键 / 值视图 \texttt{views::keys / views::values}

\begin{lstlisting}[style=C++]
map<int, string> m{{1, "a"}, {2, "b"}};
auto keys = m | views::keys;      // 1, 2
auto values = m | views::values;  // "a", "b"
\end{lstlisting}

连接 \texttt{views::join}

\begin{lstlisting}[style=C++]
std::vector<std::vector<int>> vecs{{1,2}, {3,4}};
auto flattened = vecs | views::join;  // 输出: 1 2 3 4
\end{lstlisting}

\newpage

分割字符串 \texttt{views::split}

\begin{lstlisting}[style=C++]
#include <string>
std::string str = "hello,world,cpp";
auto parts = str | views::split(',');
// parts: "hello" "world" "cpp"
\end{lstlisting}

生成序列 \texttt{views::iota}

\begin{lstlisting}[style=C++]
// 生成无限序列
auto infinite = views::iota(1);  // 1, 2, 3, 4...
auto first5 = views::iota(1) | views::take(5);  // 1, 2, 3, 4, 5

// 带步长
auto evens = views::iota(0, 10, 2);  // 0, 2, 4, 6, 8
\end{lstlisting}

滑动窗口 \texttt{views::adjacent / views::adjacent\_{}transform}

\begin{lstlisting}[style=C++]
auto pairs = nums | views::adjacent<2>;  // (1,2) (2,3) (3,4) (4,5) (5,6)
auto diffs = nums | views::adjacent_transform<2>([](int a, int b) {
    return b - a;
});  // 1 1 1 1 1
\end{lstlisting}

转换为容器

\begin{lstlisting}[style=C++]
auto view = nums | views::take(3);
vector<int> vec(view.begin(), view.end());
\end{lstlisting}

范围 for 循环中使用

\begin{lstlisting}[style=C++]
for (int n : nums | views::filter(int x { return x > 3; })) {
    std::cout << n << " ";  // 4 5 6
}
\end{lstlisting}